\documentclass[a4paper,11pt]{jreport}
\usepackage{amsmath}
\usepackage[dvips]{graphicx}
\usepackage[top=20truemm,bottom=20truemm,left=15truemm,right=15truemm]{geometry}
\usepackage{sincostan}

\title{
  高校数学の楽しい問題集
}
\author{koturn}
\date{2014年5月15日}

\begin{document}
% \maketitle
\chapter*{
  高校数学の楽しい問題集
}
\begin{flushright}
  作成者:koturn
\end{flushright}


% ============================================================================ %
\renewcommand{\thesection}{問\arabic{section}}
\renewcommand{\thesubsection}{}
\setcounter{section}{0}

\section{}
\subsection*{問題}
$x$ に関する方程式

\begin{equation}
  \sin x + \cos x + \tan x = 0 \:\:\:\: \left( -\pi < x < \pi \: \text{かつ} \: x \neq \dfrac{n \pi}{2} \: (\text{ただし,} \: n = -1, 0, 1) \right)
  \label{eq:sincostan}
\end{equation}

の実数解の個数を求めよ.

\subsection*{解答}
与えられた$x$の範囲において,$\tan x$,$\tan \dfrac{x}{2}$ は定義可能であり,
\begin{equation}
  -\dfrac{\pi}{2} < \dfrac{x}{2} < \dfrac{\pi}{2} \: \text{かつ} \: \dfrac{x}{2} \neq \pm \dfrac{\pi}{4}, 0 \: \text{であるから}\: \tan \dfrac{x}{2} \neq \pm 1, 0
\end{equation}

\begin{equation}
  -\dfrac{\pi}{2} < \dfrac{x}{2} < \dfrac{\pi}{2} \: \text{であるから}\: \cos \dfrac{x}{2} \neq 0
\end{equation}

である.ここで,$t = \tan \dfrac{x}{2}$ とおくと,$\sin x$,$\cos x$,$\tan x$は(ゼロ割りの心配無く)それぞれ,$t$を用いて,

\begin{equation}
  \sin x
  = \dfrac{2 \sin \dfrac{x}{2} \cos \dfrac{x}{2}}{1} \times 1
  = \dfrac{2 \sin \dfrac{x}{2} \cos \dfrac{x}{2}}{\cos^2 \dfrac{x}{2} + \sin^2 \dfrac{x}{2}} \times \dfrac{\dfrac{1}{\cos^2 \dfrac{x}{2}}}{\dfrac{1}{\cos^2 \dfrac{x}{2}}}
  = \dfrac{2 \tan \dfrac{x}{2}}{1 + \tan^2 \dfrac{x}{2}}
  = \dfrac{2t}{1 + t^2}
\end{equation}

\begin{equation}
  \cos x
  = \dfrac{\cos^2 \dfrac{x}{2} - \sin^2 \dfrac{x}{2}}{1} \times 1
  = \dfrac{\cos^2 \dfrac{x}{2} - \sin^2 \dfrac{x}{2}}{\cos^2 \dfrac{x}{2} + \sin^2 \dfrac{x}{2}} \times \dfrac{\dfrac{1}{\cos^2 \dfrac{x}{2}}}{\dfrac{1}{\cos^2 \dfrac{x}{2}}}
  = \dfrac{1 - \tan^2 \dfrac{x}{2}}{1 + \tan^2 \dfrac{x}{2}}
  = \dfrac{1 - t^2}{1 + t^2}
\end{equation}

\begin{equation}
  \tan x
  = \dfrac{2 \tan \dfrac{x}{2}}{1 - \tan^2 \dfrac{x}{2}}
  = \dfrac{2t}{1 - t^2}
\end{equation}

とおくことができる.これより,方程式(\ref{eq:sincostan})は,$t$に関する方程式

\begin{equation}
  \dfrac{2t}{1 + t^2} + \dfrac{1 - t^2}{1 +t^2} + \dfrac{2t}{1 - t^2} = 0
  \label{eq:sincostan_t}
\end{equation}

に置き直すことができる.
この$t$に関する方程式(\ref{eq:sincostan_t})の両辺に $(1 + t^2) (1 - t^2)$ を掛けると,

\begin{eqnarray}
  2t (1 - t^2) + (1 - t^2) + 2t (1 + t^2) & =  & 0
  \nonumber \\
  t^4 - 2t^2 + 4t + 1 & =  & 0
\end{eqnarray}

という$t$に関する4次方程式に帰着できる.

$- \pi < x < \pi$において,$t = \tan \dfrac{x}{2}$ は単調増加するので,ある$t$($\neq \pm 1, 0$)の値に対して,1つの$x$の値が対応している.
したがって,$x$についての方程式(\ref{eq:sincostan})と$t$についての方程式(\ref{eq:sincostan_t})の実数解の個数は等しいと考えられるので,方程式($\ref{eq:sincostan_t}$)の解の個数を調べる.

ここで,

\begin{equation}
  f(t) = t^4 - 2t^2 + 4t + 1
\end{equation}

とおくと,

\begin{equation}
  f'(t) = 4t^3 - 4t + 4 = 4(t^3 - t + 1)
\end{equation}

\begin{equation}
  f''(t) = 4(3t^2 - 1) = 12 \left( t + \dfrac{1}{\sqrt{3}} \right) \left( t - \dfrac{1}{\sqrt{3}} \right)
\end{equation}

となる.
しかし,方程式$f'(t) = 0$を解くのは容易ではなく,$f(t)$の増減表を描くことができない.
そこで,方程式$f''(t) = 0$を解き,$f'(t)$の増減表を描くことで,方程式$f'(t) = 0$の解について調べる.

$f'(t)$の増減表は,表\ref{tbl:second_incdec}の通り.

\begin{table}[!htbp]
  \centering
  \caption{$f'(t)$の増減表}
  \begin{tabular}{c|c|c|c|c|c|c|c|}
    $t$      & $(- \infty)$ & $\cdots$   & $- \dfrac{1}{\sqrt{3}}$                      & $\ldots$   & $\dfrac{1}{\sqrt{3}}$                          & $\ldots$   & $(+ \infty)$ \\
    \hline
    $f''(t)$ &              & $+$        & $0$                                          & $-$        & 0                                              & $+$        & \\
    \hline
    $f'(t)$  & $(- \infty)$ & $\nearrow$ & $4 \left( \dfrac{2 \sqrt{3}}{9} + 1 \right)$ & $\searrow$ & $4 \left( - \dfrac{2 \sqrt{3}}{9} + 1 \right)$ & $\nearrow$ & $(+ \infty)$ \\
  \end{tabular}
  \label{tbl:second_incdec}
\end{table}

\begin{center}
  \begin{figure}[htbp]
    \begin{center}
      \resizebox{100mm}{!}{\input{img/second_incdec}}
      \caption{$y = f'(t)$のグラフ}
    \end{center}
  \end{figure}
\end{center}

増減表\ref{tbl:second_incdec}より,$f'(\alpha) = 0$となる実数$\alpha < -\dfrac{1}{\sqrt{3}}$ がただひとつ存在する(換言すれば,$f(t)$の極値は1つだけである).

ところで,$f(t)$の増減表は,表\ref{tbl:first_incdec}に示す通りになる.

\begin{table}[!Hhtbp]
  \centering
  \caption{$f(t)$の増減表}
  \begin{tabular}{c||c|c|c|c|c}
    $t$      & $(- \infty)$ & $\cdots$   & $\alpha$    & $\ldots$   & $(+ \infty)$ \\
    \hline
    $f'(t)$  &              & $-$        & $0$         & $+$        & \\
    \hline
    $f(t)$   & $(+ \infty)$ & $\searrow$ & $f(\alpha)$ & $\nearrow$ & $(+ \infty)$
  \end{tabular}
  \label{tbl:first_incdec}
\end{table}

\begin{center}
  \begin{figure}[htbp]
    \centering
    \resizebox{100mm}{!}{\input{img/first_incdec}}
    \caption{$y = f(t)$のグラフ}
  \end{figure}
\end{center}

この増減表\ref{tbl:first_incdec}と,$\alpha < - \dfrac{1}{\sqrt{3}}$より,

\begin{eqnarray}
  f(\alpha)
  & < &
  f \left( - \dfrac{1}{\sqrt{3}} \right)
  \nonumber \\ & = &
  \left( - \dfrac{1}{\sqrt{3}} \right)^4 -2 \left( - \dfrac{1}{\sqrt{3}} \right)^2 + 4 \left( - \dfrac{1}{\sqrt{3}} \right) + 1
  \nonumber \\ & = &
  \dfrac{4}{9} \left( 1 - 3 \sqrt{3} \right)
  \nonumber \\ & < &
  0
\end{eqnarray}

したがって,$t$についての方程式(\ref{eq:sincostan_t})は2つの実数解を持ち,その実数解は明らかに$t \neq \pm 1, 0$.
ゆえに,$x$についての方程式(\ref{eq:sincostan})は2つの実数解を持つ.

\end{document}
